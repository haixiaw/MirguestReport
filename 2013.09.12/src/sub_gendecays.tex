\begin{frame}
    \begin{center}
        \LARGE GenDecay
    \end{center}
\end{frame}

\begin{frame}
    \frametitle{Introduction to GenDecay
                \footnote{DYB DocDB 3700}}
    \begin{itemize}
        \item To replace the standalone executables producing HepEvt.
            \begin{itemize}
                \item Do not integrate into offline framework
                \item Random Number Engine is different.
                \item Some implemented in FORTRAN.
            \end{itemize}
        \item Use nuclear decay data
            \begin{itemize}
                \item Evaluated Nuclear Structure Data File(ENSDF)
                        \footnote{www.nndc.bnl.gov/ensdf}.
                \item Use libmore to parse ENSDF files(But It seems the url to
                        get the dataset had been changed, I can't get the data
                        by the program.).
                \item $\alpha$, $\beta$ and $\gamma$ decays.
                \end{itemize}
        \item building a chain using {\tt NucState} and {\tt NucDecay}
            \begin{description}
                \item[{\tt NucState}] Z/A, half life, energy level, collection
                    of {\tt NucDecay}
                \item[{\tt NucDecay}] connect NucStates of mother and daughter,
                    decay type, energy of radiation and decay fraction.
            \end{description}
    \end{itemize}
\end{frame}

\begin{frame}
    \frametitle{The Design of GenDecay}
    \begin{itemize}
        \item {\tt GtDecayerator}
            \begin{itemize}
                \item A GenTool, implemented {\tt mutate}.
                \item use helper classes to generate an event in the Decay
                    Chain.
            \end{itemize}
        \item Helper Classes and Functions
            \begin{itemize}
                \item {\tt NucState}
                \item {\tt NucDecay}
                \item {\tt Radiation}, include $\alpha$ decay, $\beta^\pm$
                    decay, $\gamma$ decay, Electron Capture.
                \item Energy of radiation is generated using Monte Carlo
                    according to the spectrum.
                \item {\tt DecayRates}, generate a decay in the decay chain.
                    \footnote{still reading this code.}
            \end{itemize}
    \end{itemize}
\end{frame}

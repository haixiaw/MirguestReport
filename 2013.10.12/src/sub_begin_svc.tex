
\begin{frame}
    \frametitle{A new Service}
    \begin{itemize}
        \item In sniper, a Service derives from {\tt SvcBase} and 
                the interface of the service.
        \item Create the interface first. 
        \item Service interface will be used by other packages, so make the 
                header file public.
        \item Workflow:
        \begin{enumerate}
            \item create the service interface.
            \item create the implementation of the service.
            \item use "{\tt dynamic\_cast<IMySvc>(service(svc\_name))}"
                  to load the specific service in an algorithm.
        \end{enumerate}
    \end{itemize}
\end{frame}

% create interface
\newsavebox{\createinterface}
\begin{lrbox}{\createinterface}
\begin{lstlisting}[language=bash]
$ # first, cd to our project
$ cd $DYB2Area/tutorial/HelloAlg
$ mkdir Hello 
$ # edit Hello/IMySvc.hh
\end{lstlisting}
\end{lrbox}

% header of interface.
\newsavebox{\createinterfaceheader}
\begin{lrbox}{\createinterfaceheader}
\begin{lstlisting}
#ifndef IMySvc_hh
#define IMySvc_hh
class IMySvc {
public:
    virtual std::string what()=0;
    virtual ~IMySvc(){};
};
#endif
\end{lstlisting}
\end{lrbox}

\begin{frame}
    \frametitle{Create an interface}
    \par\usebox{\createinterface}
    \begin{block}{The interface in \(Hello/IMySvc.hh\)}
    \par\usebox{\createinterfaceheader}
    \end{block}
\end{frame}

% impl svc
% create interface
\newsavebox{\implinterfacesh}
\begin{lrbox}{\implinterfacesh}
\begin{lstlisting}[language=bash]
$ # first, cd to our project
$ cd $DYB2Area/tutorial/HelloAlg
$ # edit src/HelloSvc.hh
$ # edit src/HelloSvc.cc
$ # edit cmt/requirements

$ cd cmt
$ make
\end{lstlisting}
\end{lrbox}
% create requirement
\newsavebox{\implinterfacereq}
\begin{lrbox}{\implinterfacereq}
\begin{lstlisting}
package HelloAlg

use SniperKernel v*
use Boost v* Externals

apply_pattern install_more_includes more="Hello" 

library HelloAlgs HelloWorldAlg.cc
library HelloSvcs HelloSvc.cc
apply_pattern linker_library library=HelloAlgs
\end{lstlisting}
\end{lrbox}

\begin{frame}
    \frametitle{Implement the interface}
    \par\usebox{\implinterfacesh}
    \begin{block}{requirements}
    \par\usebox{\implinterfacereq}
    \end{block}
\end{frame}

% create header
\newsavebox{\implinterfaceheader}
\begin{lrbox}{\implinterfaceheader}
\begin{lstlisting}
#ifndef HelloSvc_hh
#define HelloSvc_hh

#include "SniperKernel/SvcBase.h"
#include "Hello/IMySvc.hh"

class HelloSvc: public SvcBase, public IMySvc {
    public:
        HelloSvc(const std::string& name);
        ~HelloSvc();
        bool initialize();
        bool finalize();
        std::string what();
    private:
        std::string m_sth;
};
#endif
\end{lstlisting}
\end{lrbox}
% create impl
\newsavebox{\implinterfaceimpl}
\begin{lrbox}{\implinterfaceimpl}
\begin{lstlisting}[linebackgroundcolor={\ifnum\value{lstnumber}=4\color{green}\fi}]
#include "HelloSvc.hh"
#include "SniperKernel/SvcFactory.h"

DECLARE_SERVICE(HelloSvc);

HelloSvc::HelloSvc(const std::string& name)
    : SvcBase(name)
{
    declProp("Something", m_sth);
}
HelloSvc::~HelloSvc()
{
}
bool HelloSvc::initialize()
{ return true; }
bool HelloSvc::finalize()
{ return true; }
std::string HelloSvc::what()
{ return m_sth; }
\end{lstlisting}
\end{lrbox}

\begin{frame}
    \frametitle{Implement the interface(Cont.)}
    \begin{itemize}
        \item The properties of service should be declared in the 
                implementation.
        \item The properties in different services with the same interface 
                can be different.
    \end{itemize}
    \begin{block}{\(src/HelloSvc.hh\)}
    \par\usebox{\implinterfaceheader}
    \end{block}
\end{frame}
\begin{frame}
    \frametitle{Implement the interface(Cont.)}
    \begin{block}{\(src/HelloSvc.cc\)}
    \par\usebox{\implinterfaceimpl}
    \end{block}
\end{frame}

% call svc from alg
\newsavebox{\callsvc}
\begin{lrbox}{\callsvc}
\begin{lstlisting}
bool
HelloWorldAlg::initialize()
{
    // ... ...
    LogInfo << "Use some service" << std::endl;
    IMySvc* ms = dynamic_cast<IMySvc*>(service(m_myName));
    LogInfo << ms->what() << std::endl;
    return true;
}
\end{lstlisting}
\end{lrbox}
\newsavebox{\callsvcpy}
\begin{lrbox}{\callsvcpy}
\begin{lstlisting}[language=python]
mgr = libSniperMgr.SniperMgr()
sp.setProperty("Sniper", "EvtMax", 5)
sp.setProperty("Sniper", "InputSvc", "NONE")
sp.setProperty("Sniper", "LogLevel", 2)
sp.setProperty("Sniper", "Dlls", ["HelloSvcs", "HelloAlgs"])

mgr.configure()

s = sp.SvcMgr.get("HelloSvc/JUNO",True)
s.setProp("Something", "JUNOSVC")
s2 = sp.SvcMgr.get("HelloSvc/JUNO2",True)
s2.setProp("Something", "Hello JUNO")

x = sp.AlgMgr.get("HelloWorldAlg/x",True)
x.setProp("myName", "JUNO2")
\end{lstlisting}
\end{lrbox}

\begin{frame}
    \frametitle{Call Service in Algorithm}
    \begin{itemize}
        \item For flexibility, use a property to select the service.
        \item Here, {\tt m\_myName} is used to select the service.
    \end{itemize}
    \begin{block}{\(src/HelloWorldAlg.cc\)}
        \par\usebox{\callsvc}
    \end{block}
\end{frame}

\begin{frame}
    \frametitle{Call Service in Algorithm(Cont.)}
    \begin{itemize}
        \item Use {\tt SvcMgr} to create or get service in python.
        \item So you can configure the service.
    \end{itemize}
    \begin{block}{\(share/job1.py(code\ fragment)\)}
        \par\usebox{\callsvcpy}
    \end{block}
\end{frame}

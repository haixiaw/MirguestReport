\section{Architecture Overview}

\subsection{DIRAC Framework}

% refer to CHEP2009, DIRAC3
DIRAC has been adopted as the job management system and data management
system for BESIII grid. It was originally started in 2002 as a tool
to support a large-scale production of Monte-Carlo experiment modelling
data for the LHCb Collaboration. The DIRAC project provides a 
complete set of tools to support the whole LHCb data processing.
It was reorganized to seperate generic and LHCb specific functionality
in 2008-2010.
% refer to Ricardo 2012-07 IHEP

DIRAC choose Python as the main programming language.
The software is well structured. 
It consists of many cooperating distributed services and light-weight
agents which deliver workload to the resources.
Modular design applied in various system components allowed to quickly
introduce new functionality and make component modifications without
affecting other components.
Its Secure client/service framework called DISET is full-featured
to implement these systems.

\subsection{The BESDIRAC}

% refer to Catriana
Like LHCb, there are BESIII specific functionalities in the grid.
The BESDIRAC is developed to provide these functionalities.
BADGER, the BESIII Advanced Data ManaGER, is the first sub-system
in BESDIRAC. It is implemented based on DIRAC File Catalog(DFC)
and integrated with DIRAC job management.
It provides File Catalog, Metadata Catalog and Dataset Catalog.

The transfer system is the second sub-system implemented based on
DIRAC framework. We create services and agents in the server side.
The scalability and flexibility of DIRAC framework make the work
easy to be done. We don't modify any DIRAC code. We just put the 
new code in our system.

Git is choosed as the version control system.
Not only because DIRAC use it. There are approximately 350
collaboration members from 52 insitutions in BESIII.
The remote sites in grid will increase and the developers for DIRAC
will also increase. So the distributed version control system 
is good at this job.

\subsection{The architecture of the Transfer System}

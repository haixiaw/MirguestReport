\begin{frame}
    \begin{center}
        \LARGE \tt{Installation}
    \end{center}
\end{frame}

\begin{frame}
    \frametitle{Mandatory Dependencies}
\begin{itemize}
    \item Below list the external libraries in JUNO offline software.
    They are required to run a simple sniper example.
    \begin{itemize}
        \item CMT, v1r26
        \item Subversion(svn), 1.6
        \item ROOT, 5.34
        \item Python, 2.7
        \item Boost, 1.53, compiled with python.
    \end{itemize}
    \item In lxslc5, you can just source the files in
    {\tt /afs/ihep.ac.cn/soft/dayabay/jmne/external/local/setup/}.
    Then you will have these software.
    \item The instruction later will tell you how to build from scratch
    in SLC5.
    \item Visit http://dyb2app1.ihep.ac.cn/mediawiki/index.php/ExternalLib
    to see details.
    \item In the future, we will use a script to install all these external
    libraries.
\end{itemize}
\end{frame}

\newsavebox{\installcmt}
\begin{lrbox}{\installcmt}
\begin{lstlisting}[language=bash]
$ wget http://www.cmtsite.net/v1r26/CMTv1r26.tar.gz
$ tar zxvf CMTv1r26.tar.gz 
$ cd CMT/v1r26/mgr/
$ ./INSTALL 
$ source setup.sh 
$ make
# now we can setup it. 
$ source setup.sh 
$ echo $CMTROOT 
..../tutProject/externals/CMT/v1r26
\end{lstlisting}
\end{lrbox}

\begin{frame}
    \frametitle{Building External Libraries}
    In SLC5, we need build all these external libraries.
    In PC like ubuntu, you can use {\tt apt-get install} to install
    libraries such as Boost.
    It is better to create a script to setup your own
    libraries quickly.
    We will use {\tt \$DYB2External} to indicate the root directory
    of the external libraries.
    \begin{block}{CMT}
        \begin{itemize}
            \item The website: http://www.cmtsite.net/
            \item Download the source kit, and use 
                {\tt tar zxvf} to extract it.
            \item Install can refer to 
                http://www.cmtsite.net/install.html.
        \end{itemize}
        \par\usebox{\installcmt}
    \end{block}
\end{frame}

% Python
\newsavebox{\installpython}
\begin{lrbox}{\installpython}
\begin{lstlisting}[language=bash]
$ # check python version first, 
$ # if it is already 2.7, just skip this slide.
$ python -V
Python 2.4.3
$ wget http://python.org/ftp/python/2.7.5/Python-2.7.5.tgz
$ tar zxvf Python-2.7.5.tgz 
$ cd Python-2.7.5
$ ./configure --prefix=$DYB2External --enable-shared
$ make -j8 && make install
$ # set the variable $PKG_CONFIG_PATH
$ export PKG_CONFIG_PATH=$DYB2External/lib/pkgconfig:$PKG_CONFIG_PATH
\end{lstlisting}
\end{lrbox}

\begin{frame}
    \frametitle{Building External Libraries(Cont.)}
    It is also necessary to have Python 2.7 installed.
    We will use python in our job configuration.

    \begin{block}{Python}
        \par\usebox{\installpython}
    \end{block}

    After Python is installed, we need to install Boost.
    Please follow the instruction in Boost. Boost Python
    is required to be compiled.
\end{frame}

% get sniper
\newsavebox{\getsniper}
\begin{lrbox}{\getsniper}
\begin{lstlisting}[language=bash]
# set the environment variable $DYB2Area first
$ echo $DYB2Area
/afs/ihep.ac.cn/users/l/lint/data/workfs/jmne/tutProject
$ cd $DYB2Area
$ svn co \
> http://dyb2app1.ihep.ac.cn/svn/sniper/branches/sniper-with-python\
>  sniper
$ export CMTPROJECTPATH=$DYB2Area:$CMTPROJECTPATH
$ echo $CMTPROJECTPATH
/afs/ihep.ac.cn/users/l/lint/data/workfs/jmne/tutProject:
\end{lstlisting}
\end{lrbox}

\begin{frame}
    \frametitle{Get Sniper}
    \begin{itemize}
        \item The sniper with python binding can be found
                in http://dyb2app1.ihep.ac.cn/svn/sniper/branches/sniper-with-python/.
        \item The development version can be found in github:
                https://github.com/mirguest/sniper
        \item Suppose you have a working directory 
                called {\tt \$DYB2Area}.
        \item Append this path into {\tt \$CMTPROJECTPATH}
    \end{itemize}
    \begin{block}{Get it}
        \par\usebox{\getsniper}
    \end{block}
\end{frame}

\begin{frame}
    \frametitle{Get Sniper(Cont.): Some Tips}
    \begin{itemize}
        \item In CMT, {\tt \$CMTPROJECTPATH}
                is used to locate the projects.
        \item So, if you want CMT to locate the 
                project sniper, you need to add the 
                parent directory of sniper into 
                the environment variable {\tt \$CMTPROJECTPATH}.

    \end{itemize}
\end{frame}

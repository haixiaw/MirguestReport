\begin{frame}
    \begin{center}
        \LARGE \tt{Configuration}
    \end{center}
\end{frame}

\begin{frame}
    \frametitle{Configure the External library}
    \begin{itemize}
        \item Some environment variables need to be
            set.
        \item For python, make sure {\tt pkg-config}
            can find python.
            \begin{itemize}
                \item {\tt pkg-config --cflags python}
                \item {\tt pkg-config --libs python}
                \item If some problems happens, check
                    the variable {\tt \$PKG\_CONFIG\_PATH}
            \end{itemize}
        \item For Boost, set variable {\tt \$BOOSTHOME}
            \begin{itemize}
                \item If you install Boost in {\tt \$DYB2External},
                    the {\tt \$BOOSTHOME} should be set to
                    {\tt \$DYB2External}.
                \item If the Boost libraries are found in 
                    {\tt /usr/lib}, the {\tt \$BOOSTHOME}
                    should be set to {\tt /usr}.
            \end{itemize}
        \item Here, the setup is complicated.
            We need to simplify this. (Please give comments.)
            \begin{itemize}
                \item Use a unified directory structure.
                \item Use unified environment variables.
                \item Separate external libraries from sniper.
                \item A good example: LCGCMT.
            \end{itemize}
    \end{itemize}
\end{frame}


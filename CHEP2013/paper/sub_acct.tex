\section{Accounting}

The accounting system in DIRAC is flexible. We only need to define 
what a data type is and how this data will be shown.
Even though there is a type called {\tt DataOperation} in DIRAC,
we define our own type. We don't want to mix the data between 
transfer system and the other systems in DIRAC.
It is also a try to have our own data type.
The schema is defined in table \ref{tb:acct}.
The value fields are grouped by the key fields.

\begin{table}
    \caption{\label{tb:acct} Schema of the data transfer accounting type}
    \begin{center}
        \begin{tabular}{lll}
            \br
            fields catalog        &   fields name   & fields type \\
            \mr
            \multirow{5}{*}{Key Fields}
                                  &   User          & VARCHAR(32) \\
                                  &   Source        & VARCHAR(32) \\
                                  &   Destination   & VARCHAR(32) \\
                                  &   Protocol      & VARCHAR(32) \\
                                  &   FinalStatus   & VARCHAR(32) \\
            \mr
            \multirow{3}{*}{Value Fields}
                                  &   TransferSize  & BIGINT UNSIGNED \\
                                  &   TransferTime  & FLOAT \\
                                  &   TransferOK    & INT UNSIGNED \\
            \br
        \end{tabular}
    \end{center}
\end{table}

%!Mode:: "TeX:UTF-8"
%\begin{frame}
%    \frametitle{OUTLINE}
%    \begin{itemize}    
%        \item
%    \end{itemize}
%\end{frame}

\begin{frame}
    \begin{center}
        \LARGE \tt{Dyb2Sim}
    \end{center}
\end{frame}

\begin{frame}
    \frametitle{软件依赖的库}
    \begin{itemize}
        \item \tt{GEANT4 9.4}
        \item \tt{ROOT 5.34}
        \item \tt{CLHEP-2.1.0.1}
        \item \tt{CERNLIB 2006}
            \begin{itemize}
                \item 主要用于产生子Thorium和Uranium
            \end{itemize}
    \end{itemize}
    \begin{alertblock}{说明}
    \begin{itemize}
        \item 关于运行代码的操作系统,理论应该支持linux的各发行版
        \item 测试过的系统有\tt{SLC},\tt{Ubuntu}和\tt{OpenSUSE}
        \item 但\tt{CERNLIB 2006}还没有迁移至\tt{OpenSUSE}
        \item 目前的代码只支持到\tt{GEANT4 9.4},未迁移至\tt{GEANT4 9.5}
    \end{itemize}
    \end{alertblock}
\end{frame}

\begin{frame}
    \frametitle{软件的目录结构}
    \begin{itemize}    
        \item \tt{DetSim}
            \begin{itemize}
                \item \tt{DetSimX}
                \item \tt{GenSim}
                \item \tt{PhysiSim}
                \item \tt{PMTSim}
                \item \tt{SimUtil}
            \end{itemize}
        \item \tt{Generator}
            \begin{itemize}
                \item \tt{InverseBeta}
                \item \tt{RadioActivity}
            \end{itemize}
    \end{itemize}
\end{frame}

\begin{frame}
    \frametitle{子模块简介}
    \begin{block}{\tt{DetSimX}}
        \begin{itemize}    
            \item 由于二期的探测器处于原型开发阶段,有很多方案
            \item 我们目前暂定:根据方案号\tt{X},创建目录\tt{DetSimX}。
            \item 目前提供\tt{DetSim0}作为参考模板。
        \end{itemize}
    \end{block}
    \begin{block}{\tt{GenSim}}
    \begin{itemize}    
        \item 主要用于\tt{GEANT4}的\tt{Primary Generator Action}
        \item 包含的功能:
            \begin{itemize}    
                \item 位置产生
                \item 解析\tt{HepEvt}
            \end{itemize}
    \end{itemize}
    \end{block}
\end{frame}

\begin{frame}
    \frametitle{子模块简介}
    \begin{block}{\tt{PhysiSim}}
        \begin{itemize}    
            \item 主要用于\tt{GEANT4}的物理过程
            \item 来源于一期的代码
        \end{itemize}
    \end{block}

    \begin{block}{\tt{PMTSim}}
        \begin{itemize}    
            \item 主要用于PMT的模拟
            \item 需要更加灵活
        \end{itemize}
    \end{block}

    \begin{block}{\tt{SimUtil}}
        \begin{itemize}    
            \item 主要用于模拟中的公用代码
        \end{itemize}
    \end{block}
\end{frame}

%!Mode:: "TeX:UTF-8"

\documentclass[20pt]{beamer}
\usepackage{xeCJK}
\usepackage{xcolor}
\usepackage{soul}
\usepackage{url}

%\setCJKmainfont{Adobe Song Std}
%\setCJKmainfont{WenQuanYi Zen Hei Mono}
\setCJKmainfont{AR PL SungtiL GB}

\usepackage{listings}
\usepackage{adjustbox}
\usepackage{ulem}

\definecolor{dkgreen}{rgb}{0,0.6,0}
\definecolor{gray}{rgb}{0.5,0.5,0.5}
\definecolor{mauve}{rgb}{0.58,0,0.82}

\setstcolor{red}
\lstset{
    language=sh,
    %basicstyle=\footnotesize\ttfamily,
    basicstyle=\scriptsize\ttfamily,
    frameround=tttt,
    frame=single,
    keywordstyle=\color{blue},
    commentstyle=\color{dkgreen},
    stringstyle=\color{mauve},
    numbers=left,
    numberstyle=\tiny\color{gray},
    numbersep=5pt,
    escapeinside={(*@}{@*)},
    %otherkeywords={$, \{, \}, \[, \]},
}
\setbeamertemplate{footline}[frame number]
\setbeamertemplate{navigation symbols}{}

\begin{document}

\title{Git + Trac + Dyb2Sim}
\author{
    \texorpdfstring{林韬
                    \newline
                    \href{mailto:lintao@ihep.ac.cn}
                    {\footnotesize\ttfamily{lintao@ihep.ac.cn}}}
                    {Lin Tao}
}

\institute{IHEP}

\maketitle

\begin{frame}
    \frametitle{OUTLINE}
    \tableofcontents
\end{frame}

%\begin{frame}
%    \frametitle{OUTLINE}
%    \begin{itemize}    
%        \item
%    \end{itemize}
%\end{frame}

\section{Introduction}

\begin{frame}
    \frametitle{简介Dyb2Sim}
    \begin{itemize}    
        \item Dyb2Sim 是什么?
            \begin{itemize}    
                \item Dyb2Sim 是基于GEANT4的探测器模拟程序。
                \item 使用 C++ 开发。
                \item 用于大亚湾二期探测器原型研究。
            \end{itemize}
    \end{itemize}
\end{frame}

\begin{frame}
    \frametitle{简介Git\footnote{\url{http://en.wikipedia.org/wiki/Git\_(software)}}}
    \begin{itemize}    
        \item Git 是什么?
            \begin{itemize}
                \item Git 是由 Linus Torvalds 开发,用于 Linux 内核开发。
                \item 分布式的版本控制和源代码管理系统。
            \end{itemize}
        \item 我们用它做什么?
            \begin{itemize}
                \item 对我们的代码进行管理。
                \item 高效地进行开发。
            \end{itemize}
    \end{itemize}
\end{frame}

\begin{frame}
    \frametitle{简介Trac\footnote{\url{http://en.wikipedia.org/wiki/Trac}}}
    \begin{itemize}    
        \item Trac 是什么?
            \begin{itemize}
                \item Trac 是基于web的项目管理和bug追踪系统。
                \item 用\tt{Python}开发。
            \end{itemize}
        \item 我们用它做什么?
            \begin{itemize}
                \item 汇报我们的进度。
                \item 报告程序中的bug。
                \item 开发者间进行交流。
            \end{itemize}
    \end{itemize}
\end{frame}

\section{Git}

    %!Mode:: "TeX:UTF-8"
%\begin{frame}
%    \frametitle{OUTLINE}
%    \begin{itemize}    
%        \item
%    \end{itemize}
%\end{frame}

\begin{frame}
    \begin{center}
        \LARGE \tt{GIT}
    \end{center}
\end{frame}

\begin{frame}
    \frametitle{Git基础}
    \begin{itemize}    
        \item 版本控制的概念
            \begin{description}
                \item[版本控制\footnote{\url{http://zh.wikipedia.org/wiki/\%E7\%89\%88\%E6\%9C\%AC\%E6\%8E\%A7\%E5\%88\%B6}}] 
                    是维护工程蓝图的标准作法,
                    能追踪工程蓝图从诞生一直到定案的过程。
                    此外,版本控制也是一种软件工程技巧,
                    借此能在软件开发的过程中,
                    确保由不同人所编辑的同一程式档案都得到同步。
            \end{description}
        \item 关于GIT
            \begin{description}
                \item[Git\footnote{\url{http://zh.wikipedia.org/wiki/Git}}] 
                    是一个由Linus Torvalds
                    为了更好地管理linux内核开发
                    而创立的分布式版本控制/软件配置管理软件。
            \end{description}
        \item 一个重要概念\footnote{\url{http://git-scm.com/book/en/Getting-Started-Git-Basics}}
            \begin{itemize}
                \item Git关心文件数据的整体是否发生变化。
                \item 其他版本管理系统则关心具体内容的差异。
            \end{itemize}
    \end{itemize}
\end{frame}

\begin{frame}
    \frametitle{git的版本控制模型}
    \includegraphics[width=10cm,keepaspectratio]{data/GitRevisionModel.png}
\end{frame}

\begin{frame}
    \frametitle{其他系统的版本控制模型}
    \includegraphics[width=10cm,keepaspectratio]{data/OtherRevisionModel.png}
\end{frame}

\begin{frame}
    \frametitle{分布式工作流程\footnote{\url{http://git-scm.com/book/en/Distributed-Git-Distributed-Workflows}}}
    \includegraphics[width=10cm,keepaspectratio]{data/GitDistributedWorkflow.png}
    \begin{block}{我们的开发模式}
        \begin{itemize}
            \item 公共的仓库,由管理员负责
            \item 个人的仓库,由开发者自己管理
            \item 如果要将个人的代码合并如公共的仓库,
                  提交Pull Request,告诉管理员仓库的url和分支名。
        \end{itemize}
    \end{block}
\end{frame}

\begin{frame}
    \frametitle{Git的数据流}
    \begin{columns}
        \column{6.0cm}
            \includegraphics[height=8cm,keepaspectratio]{data/GitDataFlowSimplified.png}
            \label{pic:DataFlow}
        \column{4.5cm}
            \begin{itemize}
                \item 远程仓库
                \item 本地仓库
                \item 工作目录
            \end{itemize}
    \end{columns}
\end{frame}




\section{Trac}

    %!Mode:: "TeX:UTF-8"
%\begin{frame}
%    \frametitle{OUTLINE}
%    \begin{itemize}    
%        \item
%    \end{itemize}
%\end{frame}

\begin{frame}
    \begin{center}
        \LARGE \tt{TRAC}
    \end{center}
\end{frame}

\begin{frame}
    \frametitle{为何用trac?}
    \begin{itemize}    
        \item 目前的地址:
        \item \url{http://dyb2app1.ihep.ac.cn/trac}
        \item 汇总各种信息
            \begin{itemize}
                \item 包括一些Wiki信息
                \item BUG的汇报
                \item 软件的计划
            \end{itemize}
        \item 便于查找相关的内容
        \item 在{\tt ticket}中,大家可以发贴进行交流
    \end{itemize}
    \begin{block}{术语及其作用}
        \begin{description}
            \item[Wiki] 各种知识和信息的表达,例如项目的整体组织情况
            \item[Ticket] 各种问题及目标的报告,可认为是一个事件
            \item[Milestone] 由Ticket组成,查看项目中子单元的进度
        \end{description}
    \end{block}
\end{frame}

\begin{frame}
    \frametitle{Wiki功能}
    \begin{itemize}    
        \item 这里的功能大家自己摸索
        \item 一些有趣的功能
            \begin{itemize}
                \item 插入代码
                \item 可支持
                    \begin{itemize}
                        \item c
                        \item cpp
                        \item python
                        \item 甚至是diff
                    \end{itemize}
                \item Workflow
            \end{itemize}
    \end{itemize}
\end{frame}

\newsavebox{\TracWikiCodeCpp}
\begin{lrbox}{\TracWikiCodeCpp}
\begin{lstlisting}[language=c++]
{{{#!cpp

class A{
};

}}}
\end{lstlisting}
\end{lrbox}

\lstdefinelanguage{diff}{
  morecomment=[f][\color{blue}]{@@},     
  morecomment=[f][\color{red}]-,         
  morecomment=[f][\color{dkgreen}]+,       
  morecomment=[f][\color{red}]{---}, 
  morecomment=[f][\color{dkgreen}]{+++},
}

\newsavebox{\TracWikiCodeDiff}
\begin{lrbox}{\TracWikiCodeDiff}
\begin{lstlisting}[language=diff]
{{{
#!diff
--- Version 55
+++ Version 56
@@ -115,8 +115,9 @@
     name='TracHelloWorld', version='1.0',
     packages=find_packages(exclude=['*.tests*']),
-    entry_points = """
-        [trac.plugins]
-        helloworld = myplugs.helloworld
-    """,
+    entry_points = {
+        'trac.plugins': [
+            'helloworld = myplugs.helloworld',
+        ],
+    },
 )
}}}
\end{lstlisting}
\end{lrbox}

\begin{frame}
    \frametitle{插入代码}
    \begin{block}{\tt{C++}代码}
        \par\usebox{\TracWikiCodeCpp}
    \end{block}
\end{frame}

\begin{frame}
    \frametitle{插入代码}
    \begin{block}{\tt{diff}代码}
        \par\usebox{\TracWikiCodeDiff}
    \end{block}
\end{frame}

\newsavebox{\TracWikiWorkflow}
\begin{lrbox}{\TracWikiWorkflow}
%\begin{adjustbox}{width=\textwidth,height=10cm,keepaspectratio}
\begin{lstlisting}
{{{#!Workflow

read_public_admin = dyb2app1 -> gitadmin
read_public_user = dyb2app1 -> user

read_users_admin = users -> gitadmin
read_users_user = users -> user

write_public_repo = gitadmin -> dyb2app1
write_users_repo = user -> users

}}}
\end{lstlisting}
%\end{adjustbox}
\end{lrbox}

\begin{frame}
    \frametitle{Wiki中Workflow的代码}
    \par\usebox{\TracWikiWorkflow}
\end{frame}

\begin{frame}
    \frametitle{Wiki中Workflow的效果}
    \includegraphics[width=10cm,keepaspectratio]{data/TracWorkflow.png}
\end{frame}


\begin{frame}
    \frametitle{Ticket的用处}
    \begin{itemize}    
        \item 向大家报告程序中的Bug
        \item 搜索有没有类似的Bug
        \item 设定下一步目标
        \item 例如,要进行一个新的探测器原型的开发
            \begin{itemize}
                \item 创建一个ticket,说明目标
                \item 得到的ticket的id
                \item 在个人仓库中建立一个相应ticket的分支
            \end{itemize}
    \end{itemize}
    \begin{block}{关于Ticket ID的妙用}
    \begin{itemize}    
        \item 例如,这个ticket的id为10。
        \item 在wiki中,可以用\tt{\#10}指向这个ticket。
        \item 在\tt{commit}的内容中,也可以
              使用\tt{\#10}。
        \item \tt{\scriptsize{git commit -am "Finish Ticket \#10."}}
        \item 完成后,就将此ticket关闭。
        \item Trac中,自动为\tt\textcolor{red}{{\sout{\#10}}} 创建了链接。
    \end{itemize}
    \end{block}
\end{frame}


\begin{frame}
    \frametitle{创建Ticket}
    \begin{itemize}    
        \item 创建Ticket时,不管是汇报bug,或者是一个新的目标,
              请尽量贴出有用信息,把问题描述清楚。
        \item 另外,好的习惯是,把这个Ticket中的信息补充完整。
        \item 例如,对于component来说,如果在开发DetSimX,
              那么就选择\tt{dyb2sim DetSimX}。
        \item 如果选项中的条目不完善,请告知管理员
        \item 具体的流程,还需要进一步讨论
    \end{itemize}
    \begin{block}{一些选项}
        \begin{description}
            \item[Type] 指定Ticket的类型,例如,{\tt Pull Request}
            \item[Component] 指定此Ticket对应哪块组件,例如{\tt DetSimX}
            \item[Milestone] 指定这个Ticket属于哪个Milestone,
                             例如{\tt "Dyb2 Detector MC"}
            \item[Version] 指定此Ticket针对的版本
        \end{description}
    \end{block}
\end{frame}

\begin{frame}
    \frametitle{Milestone(在Roadmap中)}
    \begin{itemize}    
        \item Trac中的Milestone,不是以一个时间段作为标记。
        \item 而是以完成的Ticket数做为标记。
        \item 因此,ticket中信息填全有助于我们了解项目的进展。
        \item 一个项目只要有进展,最会有新的Ticket出现,
              而又有旧的被关闭。
    \end{itemize}
\end{frame}

\begin{frame}
    \frametitle{Trac小结}
    \begin{itemize}    
        \item 遇到问题时,先搜索有没有类似的问题
        \item 如果能找到,看看是否已经修改了这个bug
        \item 如果没有,请创建新的Ticket,尽快告知大家
        \item 以Ticket的形式,记录新的工作
        \item 与大家分享你的工作进展
    \end{itemize}
\end{frame}


\section{Dyb2Sim}

    %!Mode:: "TeX:UTF-8"
%\begin{frame}
%    \frametitle{OUTLINE}
%    \begin{itemize}    
%        \item
%    \end{itemize}
%\end{frame}

\begin{frame}
    \begin{center}
        \LARGE \tt{Dyb2Sim}
    \end{center}
\end{frame}

\begin{frame}
    \frametitle{代码需要进一步研究}
    \begin{itemize}
        \item 目前的代码,由于没有研究透彻,bug肯定是存在的
        \item 例如\tt{Generator}中的产生子,就需要研究和改进
        \item 尽管目前可能主要是开发\tt{DetSimX}
        \item 但如果发现了问题,请您及时报告
    \end{itemize}
\end{frame}

\begin{frame}
    \frametitle{软件依赖的库}
    \begin{itemize}
        \item \tt{GEANT4 9.4}
        \item \tt{ROOT 5.34}
        \item \tt{CLHEP-2.1.0.1}
        \item \tt{CERNLIB 2006}
            \begin{itemize}
                \item 主要用于产生子Thorium和Uranium
            \end{itemize}
    \end{itemize}
\end{frame}

\newsavebox{\DybSimAddRemote}
\begin{lrbox}{\DybSimAddRemote}
\begin{lstlisting}
$ git remote add (*@\hl{yourlabel}@*) \
> git@dyb2app1.ihep.ac.cn:users/(*@\hl{you}@*)/Dyb2Sim
$ git push (*@\hl{yourlabel}@*) master
\end{lstlisting}
\end{lrbox}

\begin{frame}
    \frametitle{获得dyb2app1上的代码}
    \begin{itemize}
        \item 由于仍处于初始阶段,代码暂时从此处获得:
        \item {\scriptsize git clone git@dyb2app1.ihep.ac.cn:users/lintao/Dyb2Sim}
        \item 为了能够把您的代码放到服务器上:
    \end{itemize}
    \par\usebox{\DybSimAddRemote}
\end{frame}

\begin{frame}
    \frametitle{软件的目录结构}
    \begin{itemize}    
        \item \tt{DetSim}
            \begin{itemize}
                \item \tt{DetSimX}
                \item \tt{GenSim}
                \item \tt{PhysiSim}
                \item \tt{PMTSim}
                \item \tt{SimUtil}
            \end{itemize}
        \item \tt{Generator}
            \begin{itemize}
                \item \tt{InverseBeta}
                \item \tt{RadioActivity}
            \end{itemize}
    \end{itemize}
\end{frame}

\begin{frame}
    \frametitle{\tt{DetSimX}}
    \begin{itemize}    
        \item 由于二期的探测器处于原型开发阶段,有很多方案
        \item 我们目前暂定:根据方案号\tt{X},创建目录\tt{DetSimX}。
        \item 目前提供\tt{DetSim0}作为参考模板。
    \end{itemize}

\end{frame}

\begin{frame}
    \frametitle{\tt{GenSim}}
    \begin{itemize}    
        \item 主要用于\tt{GEANT4}的\tt{Primary Generator Action}
        \item 包含的功能:
            \begin{itemize}    
                \item 位置产生
                \item 解析\tt{HepEvt}
            \end{itemize}
    \end{itemize}
\end{frame}

\begin{frame}
    \frametitle{\tt{PhysiSim}}
    \begin{itemize}    
        \item 主要用于\tt{GEANT4}的物理过程
    \end{itemize}
\end{frame}

\begin{frame}
    \frametitle{\tt{PMTSim}}
    \begin{itemize}    
        \item 主要用于PMT的模拟
        \item 需要更加灵活
    \end{itemize}
\end{frame}

\begin{frame}
    \frametitle{\tt{SimUtil}}
    \begin{itemize}    
        \item 主要用于模拟中的公用代码
    \end{itemize}
\end{frame}


\section*{end}
\begin{frame}
    \begin{center}
        \LARGE Q \& A
    \end{center}
\end{frame}

\end{document}

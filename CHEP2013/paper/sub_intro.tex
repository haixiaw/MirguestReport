\section{Introduction}
The BESIII experiment at the Institute of High Energy Physics (IHEP)
of the Chinese Academy of Sciences (Beijing, China) uses the
high luminosity BEPCII double-ring electron-positron collider
to study physics in tau-charm energy region around 3.7GeV.
It has accumulated
the world's largest datasets at charm threshold.
%
The current computing and storage resources for BESIII 
experiment consist of about 4500 CPU cores, with 3 PB of disk
storage and 4 PB of tape storage at the central IHEP site.
% refer to catriana
The experiment-organized activity, mostly performed at IHEP
batch farm. The details about mass production can refer to table
\ref{tab:massprod}, which shows both simulation and reconstruction
jobs are cpu consuming.
\begin{table}[htbp]
    \caption{\label{tab:massprod}Jobs in BESIII mass production}
    \begin{center}
        \begin{tabular}{ll|l}
\br
        & Simulation        & Reconstruction \\
\mr
Events  & 50000             & 50000 \\
Input   & a few parameters  & 255-350MB rtraw + 1.5-2GB Random Trigger \\
Output  & 255-300MB rtraw   & 900MB dst \\
Time    & 14-19 hours       & 4-6 hours \\
\br
        \end{tabular}
    \end{center}
\end{table}

% refer to fabio: 
% http://indico.ihep.ac.cn/getFile.py/access?contribId=6&sessionId=1&resId=0&materialId=slides&confId=2779

When more and more processing data is accumulated,
the resource at IHEP is not able to support to
the data production of the experiment any more.
An alternative distributed solution has been considered.
The basic computing model is proposed to incorporate grid and
cloud computing resources with the existing IHEP batch farm.
In this model, most computing tasks including
the raw data processing, MC production
and analysis are performed at the central site, IHEP.
However, a fraction of MC generation will be done at
remote sites to meet the peak demand of cpu resources.
% refer to xiaomei
% http://indico.ihep.ac.cn/getFile.py/access?contribId=18&sessionId=4&resId=1&materialId=slides&confId=3161
The result of MC simulation should be transferred back
from other sites to IHEP for reconstruction
and the DST files for physics analysis should be transferred
to other sites. For future, reconstruction can also take place 
in remote sites, so the random trigger files also need to be 
transferred.

In this paper, the design and implementation of 
the Dataset-based Data Transfer System are presented.
In section 2 we describe why we use {\tt DIRAC} to create this system
and what the architecture of the transfer system is.
The transfer system kernel, the accounting and the user interface
are described in section 3.
Then in section 4, we show the result of our test.
Section 5 is devoted to conclusion and outlook.


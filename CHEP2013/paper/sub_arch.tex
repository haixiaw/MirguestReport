\section{Architecture Overview}

\subsection{DIRAC Framework}

% refer to CHEP2009, DIRAC3
{\tt DIRAC} \cite{bib:dirac3} 
has been adopted as the job management system and data management
system for BESIII grid \cite{bib:besdfc}. 
The {\tt DIRAC} project provides a 
complete set of tools to support the whole LHCb data processing.
It was reorganized to separate generic and LHCb specific functionality
in 2008-2010.
% refer to Ricardo 2012-07 IHEP
%
%DIRAC choose Python as the main programming language.
%The software is well structured. 
It consists of many cooperating distributed services and light-weight
agents which deliver workload to the resources.
Modular design applied in various system components allowed to quickly
introduce new functionality and make component modifications without
affecting other components.
Its secure client/service framework called {\tt DISET} \cite{bib:diset} 
is full-featured to implement these systems.

\subsection{The BESDIRAC Project}

% refer to Catriana
Like LHCb, there are BESIII specific functionality in the grid.
The {\tt BESDIRAC} \cite{bib:besdirac} 
is developed to provide these functionalities.
The structure of {\tt BESDIRAC} is same as {\tt DIRAC}.
The first level is the system name. The second level contains
{\tt Client}, {\tt Service}, {\tt Agent}, {\tt DB} and auxiliary classes.
Also, there are {\tt Web} directories if web portal is needed.
The {\tt DIRAC} will automatically load these extensions when 
{\tt BESDIRAC} is enabled in the {\tt DIRAC}'s configuration server.

{\tt BADGER} (BESIII Advanced Data ManaGER) is the first sub-system
in {\tt BESDIRAC}. It is implemented based on {\tt DIRAC File Catalog}(DFC)
and integrated with {\tt DIRAC} job management.
It provides File Catalog, Metadata Catalog and Dataset Catalog.
%
The transfer system is the second sub-system implemented based on
{\tt DIRAC} framework. Services and agents are deployed in the server.
The scalability and flexibility of {\tt DIRAC} framework make the work
easy to be done. 

\subsection{The architecture of the Transfer System}

The transfer system contains:

\begin{itemize}
    \item {\tt Transfer Agent} is the scheduler to manage transfer workers.
    \item {\tt Transfer Request Service} is to manage the transfer requests
          created by users.
    \item {\tt Transfer DB} is the shared memory between the agent and 
          service. 
    \item {\tt Dataset Service} is for dataset management which is 
          {\em deprecated} because it is in {\tt DFC} now.
    \item {\tt Accounting} is to keep the transfer history.
          It follows the structure in {\tt DIRAC} Accounting System
          \cite{bib:diracacct}.
          It can be shown in the web portal.
    \item {\tt Web Portal} and {\tt command line scripts} are the 
          User Interface for users.
\end{itemize}

The workflow is shown in Figure \ref{fig:workflow} and described as follows.
\begin{figure}[htbp]
    %\centering
    % Define block styles
\begin{center}
\tikzstyle{block} = [rectangle, draw, fill=blue!20, text width=4em, text centered, rounded corners]
\tikzstyle{hugeBlock} = [rectangle, draw, fill=blue!20,
    text width=5em, text centered, rounded corners, minimum height=3em]
\tikzstyle{line} = [draw, -latex, line width=.1em]

%\adjustbox{max width=\textwidth}{
\begin{tikzpicture}[node distance = 2cm, auto]
    \node [block] (user) {User};
    \node [block, right of=user, node distance=4cm] (transReqSvc) {Transfer Request Service};
    \node [block, below of=user, node distance=1.4cm] (DFC) {DFC};
    \node [block, right of=DFC, node distance=4cm] (datasetSvc) {Dataset Service};
    \node [block, right of=transReqSvc, node distance=4cm, yshift=-0.7cm, 
            text width=6em]
                                                    (Database) {TransferDB};
    \node [block, right of=Database, node distance=4cm] (transAgent) { Transfer Agent};
    \path[line,<->, postaction={
                    decorate,
                    decoration={
                        raise=1ex,
                        text along path,
                        text align=center,
                        text={2},
                    }
        }](user) -- (transReqSvc);
    \path[line,->, postaction={
                    decorate,
                    decoration={
                        raise=1ex,
                        text along path,
                        text align=center,
                        text={1},
                    }
        }] (user) -- (DFC);
    \path[line,->](DFC) -- (datasetSvc);
    \path[line,<->](transReqSvc) -- (Database);
    \path[line,<->](datasetSvc) -- (Database);
    \path[line,<->,postaction={
                    decorate,
                    decoration={
                        raise=1ex,
                        text along path,
                        text align=center,
                        text={3},
                    }
        }](Database) -- (transAgent);
\end{tikzpicture}
%} % adjustbox
\end{center}

    \caption{Workflow of transfer system} \label{fig:workflow}
\end{figure}

\begin{enumerate}
\item User create a dataset in DFC first. The dataset service
    will save the dataset with a list of Logical File Names.
\item User create the transfer request
whose information will be saved in the transfer DB via
the transfer request service.
\item Transfer agent will transfer these files in DB.
\end{enumerate}


\begin{frame}
    \begin{center}
        \LARGE 江门中微子实验探测器模拟及离线软件
    \end{center}
\end{frame}

\begin{frame}
    \frametitle{江门中微子实验软件工作}
    \begin{itemize}
        \item 离线软件环境的维护。
        \item 探测器模拟软件的开发。
            \begin{itemize}
                \item 为应对不同方案的探测器构建,
                      解耦PMT的构建与PMT的摆放,
                      采用PMTManager方式构建PMT。
                \item 基于PMTManager,实现球形PMT的构建,
                      并研究PMT前方有挡块(四边形/六边形)
                      时本底对能量分辨率的影响。
                \item 研究在任意Physical Volume中的粒子产生。
                      包括粒子的位置和动量的全局/局部变换。
            \end{itemize}
        \item Sniper离线框架中的Python Binding。
            \begin{itemize}
                \item 主要研究了Gaudi与BASF2框架中作业配置的实现
                \item 学习了C++模板元编程技术
                \item 实现了两种方式的Python作业配置
                    \begin{itemize}
                        \item 基于Job Option Parser
                        \item 不依赖Job Option Parser
                    \end{itemize}
                \item 修改Sniper Kernel,去除Job Option的方式,
                      改用全python的作业配置。
            \end{itemize}
    \end{itemize}
\end{frame}

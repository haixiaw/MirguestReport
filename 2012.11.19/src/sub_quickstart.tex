%!Mode:: "TeX:UTF-8"
%\begin{frame}
%    \frametitle{OUTLINE}
%    \begin{itemize}    
%        \item
%    \end{itemize}
%\end{frame}

\begin{frame}
    \begin{center}
        \LARGE \tt{Quick Start}
    \end{center}
\end{frame}

\newsavebox{\QuickStartEnvSetup}
\begin{lrbox}{\QuickStartEnvSetup}
\begin{lstlisting}
Login lxslc5.ihep.ac.cn
$ mkdir ~/tutProject
$ cd ~/tutProject
create bashrc which contains Line 6 to 8
$ cat bashrc 
source /panfs/panfs.ihep.ac.cn/home/dyb/dybsw/dyb2/external/local/bashrc
export G4WORKDIR=~/geant4
export PATH=${G4WORKDIR}/bin/Linux-g++:$PATH
$ source bashrc
\end{lstlisting}
\end{lrbox}

\newsavebox{\QuickStartGitClone}
\begin{lrbox}{\QuickStartGitClone}
\begin{lstlisting}
$ git clone (*@{\color{red}-o ltrepo}@*) git@dyb2app1.ihep.ac.cn:users/lintao/Dyb2Sim
\end{lstlisting}
\end{lrbox}

\begin{frame}
    \frametitle{环境设置 \&\& clone代码}
    \begin{block}{环境设置}
        \par\usebox{\QuickStartEnvSetup}
    \end{block}
    \begin{block}{导出代码}
        \par\usebox{\QuickStartGitClone}
    \end{block}
    \begin{alertblock}{提示}
        \begin{itemize}    
            \item {\tt -o ltrepo}是给远程仓库设定别名,因此,名字可任意
        \end{itemize}
    \end{alertblock}
\end{frame}

\newsavebox{\QuickStartDybCompile}
\begin{lrbox}{\QuickStartDybCompile}
\begin{lstlisting}
$ cd Dyb2Sim/DetSim/DetSim0
$ make
\end{lstlisting}
\end{lrbox}

\newsavebox{\QuickStartDybRun}
\begin{lrbox}{\QuickStartDybRun}
\begin{lstlisting}
$ dyb2main run.mac
\end{lstlisting}
\end{lrbox}

\newsavebox{\QuickStartDybRunMore}
\begin{lrbox}{\QuickStartDybRunMore}
\begin{lstlisting}
$ cd mac/
$ ls
\end{lstlisting}
\end{lrbox}

\begin{frame}
    \frametitle{编译 \&\& 运行}
    \begin{block}{编译}
        \par\usebox{\QuickStartDybCompile}
    \end{block}
    \begin{block}{运行}
        \par\usebox{\QuickStartDybRun}
    \end{block}
    \begin{block}{运行更多的mac文件}
        \par\usebox{\QuickStartDybRunMore}
    \end{block}
\end{frame}

\newsavebox{\QuickStartDybGeneratorCompile}
\begin{lrbox}{\QuickStartDybGeneratorCompile}
\begin{lstlisting}
$ pwd
(*@{\color{red}/afs/ihep.ac.cn/users/l/lint/}@*)tutProject/Dyb2Sim/DetSim/DetSim0/mac
$ cd ../../../Generator/InverseBeta/
$ make
$ ls
GNUmakefile  (*@{\color{green}IBD}@*)  include  src
\end{lstlisting}
\end{lrbox}

\newsavebox{\QuickStartDybGeneratorRun}
\begin{lrbox}{\QuickStartDybGeneratorRun}
\begin{lstlisting}
$ ./IBD --help
查看帮助
$ ./IBD -o (*@{\color{red}ibd.asc}@*) -n 1000
$ cp ibd.asc ../../DetSim/DetSim0/mac/
$ cd ../../DetSim/DetSim0/mac/
$ head run_hepevt_ibd.mac -n 3
/dyb2/gen/source/reset

/dyb2/gen/source/HepEvt/append (*@{\color{red}ibd.asc}@*)
$ dyb2main run_hepevt_ibd.mac 
\end{lstlisting}
\end{lrbox}

\begin{frame}
    \frametitle{编译 \&\& 运行 产生子}
    \begin{block}{编译}
        \par\usebox{\QuickStartDybGeneratorCompile}
    \end{block}
    \begin{block}{运行}
        \par\usebox{\QuickStartDybGeneratorRun}
    \end{block}
\end{frame}

\newsavebox{\QuickStartDybDevRemote}
\begin{lrbox}{\QuickStartDybDevRemote}
\begin{lstlisting}
$ git remote show 
(*@{\color{red}ltrepo}@*)
$ git remote add (*@{\color{red}ownrepo}@*) git@dyb2app1.ihep.ac.cn:users/(*@{\color{red}lintao}@*)/(*@{\color{red}tutProject}@*)
$ git remote show 
(*@{\color{red}ltrepo}@*)
(*@{\color{red}ownrepo}@*)
\end{lstlisting}
\end{lrbox}

\begin{frame}
    \frametitle{开发}
    \begin{itemize}    
        \item 根据不同的方案,建立不同的\tt{DetSim{\color{red}X}}
        \item 暂时将\tt{DetSim0}作为开发的模板
        \item 开发一个新任务时,尽量创建新的分支
        \item 把自己本地代码,push到git服务器
    \end{itemize}
\end{frame}

\begin{frame}
    \frametitle{设置远程仓库}
    \begin{itemize}    
        \item 此处,讲解如何添加远程仓库
        \item 目前的策略:每个人可以往自己的目录下{\Large 写入}或{\Large 添加}项目
        \item 其他开发者可以{\Large 读取}你的项目
    \end{itemize}
    \begin{block}{示例}
        \par\usebox{\QuickStartDybDevRemote}
    \end{block}
    \begin{alertblock}{提示}
        \begin{itemize}    
            \item {\tt ownrepo}是给远程仓库设定别名
            \item {\tt lintao}是申请时的账户
            \item {\tt tutProject}是项目的名称
        \end{itemize}
    \end{alertblock}
\end{frame}

\newsavebox{\QuickStartDybDevBranch}
\begin{lrbox}{\QuickStartDybDevBranch}
\begin{lstlisting}
$ git branch 
* master
$ git checkout -b (*@{\color{red}lintao-dev}@*)
Switched to a new branch '(*@{\color{red}lintao-dev}@*)'
$ git branch 
* (*@{\color{red}lintao-dev}@*)
  master
\end{lstlisting}
\end{lrbox}

\begin{frame}
    \frametitle{创建新的分支}
    \begin{itemize}    
        \item 分支的创建是为了不影响其他人的开发
        \item 保留master分支,便于合并
        \item 创建自己的dev分支,此分支应该push到服务器
        \item 创建其他分支时,最好从自己的dev分支开始
    \end{itemize}
    \begin{block}{示例}
        \par\usebox{\QuickStartDybDevBranch}
    \end{block}
    \begin{alertblock}{提示}
        \begin{itemize}    
            \item {\tt git checkout -b}表示导出一个新的branch
            \item 如果分支已经存在,只要{\tt git checkout}
        \end{itemize}
    \end{alertblock}
\end{frame}

\newsavebox{\QuickStartDybDevBranchPush}
\begin{lrbox}{\QuickStartDybDevBranchPush}
\begin{lstlisting}
$ git push (*@{\color{red}ownrepo}@*) (*@{\color{red}lintao-dev}@*)
... skip ...
To git@dyb2app1.ihep.ac.cn:users/lintao/tutProject
 * [new branch]      lintao-dev -> lintao-dev
\end{lstlisting}
\end{lrbox}

\newsavebox{\QuickStartDybDevBranchFetch}
\begin{lrbox}{\QuickStartDybDevBranchFetch}
\begin{lstlisting}
$ git fetch (*@{\color{red}ltrepo}@*)
\end{lstlisting}
\end{lrbox}

\begin{frame}
    \frametitle{更多关于分支的常用操作}
    \begin{block}{push分支至服务器}
        \par\usebox{\QuickStartDybDevBranchPush}
    \end{block}
    \begin{block}{同步远程服务器仓库}
        \par\usebox{\QuickStartDybDevBranchFetch}
    \end{block}
    \begin{alertblock}{提示}
        \begin{itemize}    
            \item {\tt git push}表示把当前的分支{\color{red}lintao-dev}推送到仓库{\color{red}ownrepo}
            \item 对于不想分享的分支,就不要推送了
            \item {\tt git fetch},是为了更新远程仓库的本地备份,并与{\color{red}ltrepo}同步
            \item 在多开发者的情况下,添加多个remote,可以方便同步
        \end{itemize}
    \end{alertblock}
\end{frame}

\newsavebox{\QuickStartDybDevCommit}
\begin{lrbox}{\QuickStartDybDevCommit}
\begin{lstlisting}
$ git commit -am '(*@{\color{red}Comment}@*)' 
\end{lstlisting}
\end{lrbox}

\newsavebox{\QuickStartDybDevStatus}
\begin{lrbox}{\QuickStartDybDevStatus}
\begin{lstlisting}
$ git status
$ git diff
$ gitk
\end{lstlisting}
\end{lrbox}

\begin{frame}
    \frametitle{本地工作目录常用命令}
    \begin{block}{记录每次更新到本地仓库}
        \par\usebox{\QuickStartDybDevCommit}
    \end{block}
    \begin{block}{查看工作目录的改动}
        \par\usebox{\QuickStartDybDevStatus}
    \end{block}
    \begin{alertblock}{提示}
        \begin{itemize}    
            \item 在开发过程中,本页是最常用的命令
            \item 最好让自己的代码处于跟踪状态
            \item 另外,如果有图形X,可以使用\tt{gitk}查看历史
        \end{itemize}
    \end{alertblock}
\end{frame}

\newsavebox{\QuickStartDybDevPull}
\begin{lrbox}{\QuickStartDybDevPull}
\begin{lstlisting}
$ git pull (*@{\color{red}ownrepo}@*) (*@{\color{red}lintao-dev}@*)
\end{lstlisting}
\end{lrbox}

\begin{frame}
    \frametitle{同步仓库}
    \begin{block}{使用\tt{git pull}进行同步合并}
        \par\usebox{\QuickStartDybDevPull}
    \end{block}
    \begin{alertblock}{提示}
        \begin{itemize}    
            \item \tt{git pull}可认为是\tt{git fetch}后再\tt{git merge}
            \item 即把远程仓库的分支合并到了本地仓库的相应分支中
            \item 对于同步你个人的仓库,可以使用\tt{git pull}简化操作
                \begin{itemize}
                    \item 多份指向同一远程仓库的工作目录
                \end{itemize}
        \end{itemize}
    \end{alertblock}
\end{frame}

\newsavebox{\QuickStartDybDevMerge}
\begin{lrbox}{\QuickStartDybDevMerge}
\begin{lstlisting}
$ git branch 
* lintao-dev
  master
$ git checkout -b lintao-tut-1
$ git checkout -b lintao-tut-2
$ git branch 
  lintao-dev
  lintao-tut-1
* lintao-tut-2
  master
\end{lstlisting}
\end{lrbox}

\begin{frame}
    \frametitle{分支的合并}
    \begin{block}{使用\tt{git checkout -b}创建分支}
        \par\usebox{\QuickStartDybDevMerge}
    \end{block}
    \begin{alertblock}{提示}
        \begin{itemize}    
            \item \tt{git branch}可以查看分支的状况
        \end{itemize}
    \end{alertblock}
\end{frame}

\newsavebox{\QuickStartDybDevMergeCont}
\begin{lrbox}{\QuickStartDybDevMergeCont}
\begin{lstlisting}
$ touch TUT_TMP
$ ls
DetSim  Generator  README  (*@{\color{red}TUT\_TMP}@*)
$ git add TUT_TMP
$ git commit -am 'Add TUT_TMP in lintao-tut-2.'
$ git checkout lintao-tut-1 
Switched to branch 'lintao-tut-1'
$ ls
DetSim  Generator  README
$ git merge lintao-tut-2 
$ ls
DetSim  Generator  README  (*@{\color{red}TUT\_TMP}@*)
\end{lstlisting}
\end{lrbox}

\begin{frame}
    \frametitle{分支的合并}
    \begin{block}{使用\tt{git merge}进行合并}
        \par\usebox{\QuickStartDybDevMergeCont}
    \end{block}
    \begin{alertblock}{提示}
        \begin{itemize}    
            \item 不同的分支互不影响
            \item \tt{git merge}可在不同的分支间进行合并
        \end{itemize}
    \end{alertblock}
\end{frame}

\newsavebox{\QuickStartSummary}
\begin{lrbox}{\QuickStartSummary}
\begin{lstlisting}
$ dyb2main run.mac
$ ./IBD -o (*@{\color{red}ibd.asc}@*) -n 1000
$ git clone (*@{\color{red}-o ltrepo}@*) git@dyb2app1.ihep.ac.cn:users/lintao/Dyb2Sim
$ git remote show 
$ git remote add (*@{\color{red}ownrepo}@*) git@dyb2app1.ihep.ac.cn:users/(*@{\color{red}lintao}@*)/(*@{\color{red}tutProject}@*)
$ git checkout -b (*@{\color{red}lintao-dev}@*)
$ git push (*@{\color{red}ownrepo}@*) (*@{\color{red}lintao-dev}@*)
$ git pull (*@{\color{red}ownrepo}@*) (*@{\color{red}lintao-dev}@*)
$ git commit -am '(*@{\color{red}Comment}@*)' 
$ git status
$ git diff
$ git merge lintao-tut-2 
$ gitk
\end{lstlisting}
\end{lrbox}

\begin{frame}
    \frametitle{小结}
        \begin{itemize}
            \item 本节主要涉及了一些常用的操作
            \item 涉及最多的便是\tt{git}这个工具
            \item 希望能够帮助大家把程序运行起来
            \item 并了解开发时如何使用这个工具
        \end{itemize}
        \begin{block}{常用命令回顾}
            \par\usebox{\QuickStartSummary}
        \end{block}
\end{frame}

\section{Architecture Overview}

\subsection{DIRAC Framework}

% refer to CHEP2009, DIRAC3
{\tt DIRAC} \cite{bib:dirac3,bib:diracgit} 
has been adopted as the job management system and data management
system for the BESIII grid \cite{bib:besdfc}, since 
% refer to Ricardo 2012-07 IHEP
it is a complete grid solution for a community of users such as
LHCb collaboration.
It consists of many cooperating distributed services and light-weight
agents within the same {\tt DISET} framework following the grid
security standards.
Modular design applied in various system components allowed to quickly
introduce new functionality and make component modifications without
affecting other components.

\subsection{The BESDIRAC Project}

% refer to Catriana
Like the LHCb, the BESIII also needs experiment-specific 
functionalities in its grid.
The {\tt BESDIRAC} \cite{bib:besdirac} 
is developed to provide these functionalities
based on the underlying framework of DIRAC.
The first level is the system name. The second level contains
{\tt Client}, {\tt Service}, {\tt Agent}, {\tt DB} and auxiliary classes.
Also, there are {\tt Web} directories if web portal is needed.
The {\tt DIRAC} will automatically load these extensions when 
{\tt BESDIRAC} is enabled by the {\tt DIRAC}'s configuration service.

The {\tt BADGER} (\verb"BESIII Advanced Data ManaGER") is the first sub-system
in {\tt BESDIRAC}, which is implemented using {\tt DIRAC File
Catalog}(\verb"DFC")
and integrated with the {\tt DIRAC} job management.
It provides file catalog, metadata catalog and dataset catalog.
%
The transfer system is the second sub-system implemented in the
{\tt DIRAC} framework. 
New services and agents can be easily deployed in the server
due to {\tt DIRAC}'s good scalability and flexibility.

\subsection{The architecture of the Transfer System}
The main components of the transfer system are as follows:
\begin{itemize}
    \item {\tt Transfer Agent} is the scheduler to manage transfer workers.
    \item {\tt Transfer Request Service} is to manage the transfer requests
          created by users.
    \item {\tt Transfer DB} is the shared memory between the agent and 
          service. 
    \item {\tt Dataset Service} is the dataset manager which is 
          {\em deprecated} because it is in {\tt DFC} now.
    \item {\tt Accounting} is to keep the transfer history.
          It follows the structure in {\tt DIRAC} Accounting System
          \cite{bib:diracacct}.
          It can be easily integrated with the web portal.
    \item {\tt Web Portal} and {\tt command line scripts} are the 
          two types of User Interface.
\end{itemize}

The workflow is described as follows.
\begin{enumerate}
\item Administrator creates datasets in {\tt DFC} first. 
\item User creates a transfer request of dataset using the command line 
      or the web portal.
\item The {\tt transfer request service} will save the transfer 
      request in the {\tt transfer DB}.
\item {\tt Transfer agent} will transfer these files in the dataset.
\end{enumerate}


\section{Architecture Overview}

\subsection{DIRAC Framework}

% refer to CHEP2009, DIRAC3
DIRAC has been adopted as the job management system and data management
system for BESIII grid. It was originally started in 2002 as a tool
to support a large-scale production of Monte-Carlo experiment modelling
data for the LHCb Collaboration. The DIRAC project provides a 
complete set of tools to support the whole LHCb data processing.
It was reorganized to seperate generic and LHCb specific functionality
in 2008-2010.
% refer to Ricardo 2012-07 IHEP

DIRAC choose Python as the main programming language.
The software is well structured. 
It consists of many cooperating distributed services and light-weight
agents which deliver workload to the resources.
Modular design applied in various system components allowed to quickly
introduce new functionality and make component modifications without
affecting other components.
Its Secure client/service framework called DISET is full-featured
to implement these systems.

\subsection{The architecture of the Transfer System}

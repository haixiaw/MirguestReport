\begin{frame}
    \begin{center}
        \LARGE \tt{Begin a Project}
    \end{center}
\end{frame}

% create a new project
\newsavebox{\createproject}
\begin{lrbox}{\createproject}
\begin{lstlisting}[language=bash]
$ cd $DYB2Area
$ cmt create_project tutorial
$ # edit tutorial/cmt/project.cmt
$ cat tutorial/cmt/project.cmt
project tutorial
(*@\textcolor{red}{\textbf{use sniper}}@*)
\end{lstlisting}
\end{lrbox}

\begin{frame}
    \frametitle{Begin a Project}
    \begin{itemize}
        \item We use CMT to manage our projects.
        \item CMT will locate these projects via {\tt \$CMTPROJECTPATH}.
        \item For simplicity, we can place the
               new project in {\tt \$DYB2Area}.
        \item The project should depend on sniper.
    \end{itemize}
    \begin{block}{Create A Project}
        \par\usebox{\createproject}
    \end{block}
\end{frame}

% create a new package
\newsavebox{\createpackage}
\begin{lrbox}{\createpackage}
\begin{lstlisting}[language=bash]
$ # first, cd to our project
$ cd $DYB2Area/tutorial
$ # create a package
$ cmt create HelloAlg v0
$ # Then we can write a new Alg.
$ cd HelloAlg
\end{lstlisting}
\end{lrbox}

\begin{frame}
    \frametitle{A new Package}
    \begin{itemize}
        \item After a project is created, we can create new packages
                in this project.
    \end{itemize}
    \begin{block}{Create A Package}
        \par\usebox{\createpackage}
    \end{block}
\end{frame}

\begin{frame}
    \frametitle{Package Structure}
    \begin{itemize}
        \item In a package, there are several directories and files.
        \item cmt
            \begin{itemize}
                \item {\tt requirements}, used to generate setup.*sh and Makefile.
                \item {\tt setup.*sh}, setup the runtime environment.
                \item {\tt Makefile}, compile and install the package.
            \end{itemize}
        \item {\tt src}, source code for the package, including header 
                and implementation.
        \item {\tt binding}, binding code for the package.
        \item Directory for header files, which is used for the public API.
                It means these header files will be used by 
                some other packages.
    \end{itemize}
\end{frame}

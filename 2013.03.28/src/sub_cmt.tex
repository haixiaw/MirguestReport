%!Mode:: "TeX:UTF-8"

\begin{frame}
    \begin{center}
        \LARGE \tt{CMT\footnote{本人对CMT的使用经验不够丰富,后面都是个人的见解}}
    \end{center}
\end{frame}

\begin{frame}
    \frametitle{CMT简介}
    \begin{itemize}    
        \item 
            \begin{quote}
                This environment, based on some management conventions and
                comprising several utilities, is an attempt to formalize
                software production and especially configuration management
                around a package-oriented principle.\footnote{CMTDoc.html}
            \end{quote} 
        \item 从使用角度考虑,CMT主要为我们做了两件事:
            \begin{itemize}
                \item 根据包之间的依赖关系,编译软件包
                \item 设置软件包运行时环境
            \end{itemize}
    \end{itemize}
\end{frame}

\begin{frame}
    \frametitle{CMT中的术语}
    \begin{description}
        \item[Project] 这个软件项目。包含相关的Package。
        \item[Package] CMT中软件的基本单元。我们可以对其赋予不同的语义。
        \item[Sub-Project] 由Package组成。其语义是用于对Package的归类。
    \end{description}
    \begin{block}{不同语义的Package}
        \begin{description}
            \item[Primary Package]
            \item[Policy Package]
            \item[Container or Management Package]
            \item[Release Package]
            \item[Glue or Interface Package]
        \end{description}
    \end{block}
\end{frame}

\begin{frame}
    \frametitle{新的离线软件是否选择使用CMT}
    \begin{itemize}
        \item 从BOSS到NuWa,离线软件都使用CMT进行包管理。
        \item 很多外部库都已经有了对应的CMT Requirement文件。例如LCGCMT。
        \item 用户使用方便。不需要了解Makefile的编写。
              软件运行时环境也自动设置。
        \item 可能很多人都有使用经验。已有了用户习惯。
        \item 有人给出,CMT的效率不高,相比于cmake,更加复杂。
              Gaudi除了CMT编译外,还提供了cmake的方式。
              \url{http://indico.cern.ch/getFile.py/access?contribId=2\&resId=1\&materialId=slides\&confId=105778}
    \end{itemize}
\end{frame}

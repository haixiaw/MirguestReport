\section{Introduction}
The BESIII experiment at the BEPCII collider has accumulated
the world's largest datasets at charm threshold.
%
The currently computing and storage resources for BESIII 
experiment consist of about 3400 CPU cores, with 2.4 PB of disk
storage and 4 PB of tape storage at the central IHEP site.
% refer to catriana
The experiment-organized activity, mostly performed at IHEP
batch farm. For mass production of simulated events,
each batch job generates a single file containing
50000 events. It takes about 14 to 19 hours. The output
is a ROOT file whose size is about 255 to 350 MB.
For the rectronstruction of simulated RAW events,
it takes about 4 to 6 hours. The input files are
250MB to 350MB simulated events and 1.5GB to 2GB random trigger.
The output file is a 900MB ROOT file.
The resource at IHEP is not enough.
An alternative distributed solution is required.
% refer to fabio: 
% http://indico.ihep.ac.cn/getFile.py/access?contribId=6&sessionId=1&resId=0&materialId=slides&confId=2779

The basic computing model is proposed to incorporate grid and
cloud computing resources with the existing IHEP batch farm.
IHEP central site do the raw data processing, MC production
and analysis. The remote sites do part of MC production
and analysis at peak time.
% refer to xiaomei
% http://indico.ihep.ac.cn/getFile.py/access?contribId=18&sessionId=4&resId=1&materialId=slides&confId=3161
The result of MC Simulation should be transferred back
from other sites to IHEP for reconstruction
and the DST files for physics analysis should be transferred
to other sites. For future, reconstruction also can take place 
in remote sites, so the random trigger files also need to be 
transferred.

%!Mode:: "TeX:UTF-8"
%\begin{frame}
%    \frametitle{OUTLINE}
%    \begin{itemize}    
%        \item
%    \end{itemize}
%\end{frame}

\begin{frame}
    \begin{center}
        \LARGE \tt{Dyb2Sim}
    \end{center}
\end{frame}

\begin{frame}
    \frametitle{代码需要进一步研究}
    \begin{itemize}
        \item 目前的代码,由于没有研究透彻,bug肯定是存在的
        \item 例如\tt{Generator}中的产生子,就需要研究和改进
        \item 尽管目前可能主要是开发\tt{DetSimX}
        \item 但如果发现了问题,请您及时报告
    \end{itemize}
\end{frame}

\begin{frame}
    \frametitle{软件依赖的库}
    \begin{itemize}
        \item \tt{GEANT4 9.4}
        \item \tt{ROOT 5.34}
        \item \tt{CLHEP-2.1.0.1}
        \item \tt{CERNLIB 2006}
            \begin{itemize}
                \item 主要用于产生子Thorium和Uranium
            \end{itemize}
    \end{itemize}
\end{frame}

\newsavebox{\DybSimAddRemote}
\begin{lrbox}{\DybSimAddRemote}
\begin{lstlisting}
$ git remote add (*@\hl{yourlabel}@*) \
> git@dyb2app1.ihep.ac.cn:users/(*@\hl{you}@*)/Dyb2Sim
$ git push (*@\hl{yourlabel}@*) master
\end{lstlisting}
\end{lrbox}

\begin{frame}
    \frametitle{获得dyb2app1上的代码}
    \begin{itemize}
        \item 由于仍处于初始阶段,代码暂时从此处获得:
        \item {\scriptsize git clone git@dyb2app1.ihep.ac.cn:users/lintao/Dyb2Sim}
        \item 为了能够把您的代码放到服务器上:
    \end{itemize}
    \par\usebox{\DybSimAddRemote}
\end{frame}

\begin{frame}
    \frametitle{软件的目录结构}
    \begin{itemize}    
        \item \tt{DetSim}
            \begin{itemize}
                \item \tt{DetSimX}
                \item \tt{GenSim}
                \item \tt{PhysiSim}
                \item \tt{PMTSim}
                \item \tt{SimUtil}
            \end{itemize}
        \item \tt{Generator}
            \begin{itemize}
                \item \tt{InverseBeta}
                \item \tt{RadioActivity}
            \end{itemize}
    \end{itemize}
\end{frame}

\begin{frame}
    \frametitle{\tt{DetSimX}}
    \begin{itemize}    
        \item 由于二期的探测器处于原型开发阶段,有很多方案
        \item 我们目前暂定:根据方案号\tt{X},创建目录\tt{DetSimX}。
        \item 目前提供\tt{DetSim0}作为参考模板。
    \end{itemize}

\end{frame}

\begin{frame}
    \frametitle{\tt{GenSim}}
    \begin{itemize}    
        \item 主要用于\tt{GEANT4}的\tt{Primary Generator Action}
        \item 包含的功能:
            \begin{itemize}    
                \item 位置产生
                \item 解析\tt{HepEvt}
            \end{itemize}
    \end{itemize}
\end{frame}

\begin{frame}
    \frametitle{\tt{PhysiSim}}
    \begin{itemize}    
        \item 主要用于\tt{GEANT4}的物理过程
    \end{itemize}
\end{frame}

\begin{frame}
    \frametitle{\tt{PMTSim}}
    \begin{itemize}    
        \item 主要用于PMT的模拟
        \item 需要更加灵活
    \end{itemize}
\end{frame}

\begin{frame}
    \frametitle{\tt{SimUtil}}
    \begin{itemize}    
        \item 主要用于模拟中的公用代码
    \end{itemize}
\end{frame}


\begin{frame}
    \begin{center}
        \LARGE 学习进展
    \end{center}
\end{frame}

\begin{frame}
    \frametitle{学习知识点}
    \begin{itemize}
        \item 2012年7月入室。参加{\tt DIRAC}的培训。
            \begin{itemize}
                \item 接触{\tt git} 和 {\tt github}。初步了解软件的代码管理。
                \item 研究了{\tt DIRAC}源码中的Data Management。
            \end{itemize}
        \item 接触大亚湾二期中心探测器模拟代码。
            \begin{itemize}
                \item 了解{\tt GEANT4}。研究了一些技巧,如
                    \begin{itemize}
                        \item 反射光锥运行时改变几何大小;
                        \item 特定材料上本底产生。
                    \end{itemize}
                \item 光锥的模拟:平面,柱形
                \item 二期中心探测器方案六:球形钢罐+液闪
                \item 能量分辨率的计算
            \end{itemize}
        \item 半年来阅读的图书:
            \begin{itemize}
                \item UNIX编程艺术
                \item C和指针
                \item C专家编程
                \item 设计模式(在读)
            \end{itemize}
    \end{itemize}
\end{frame}

%!Mode:: "TeX:UTF-8"
\begin{frame}
    \begin{center}
        \LARGE \tt{GIT与CMT的结合}
    \end{center}
\end{frame}
%\begin{frame}
%    \frametitle{OUTLINE}
%    \begin{itemize}    
%        \item
%    \end{itemize}
%\end{frame}

\begin{frame}
    \frametitle{Git与CMT结合时的一些问题}
    \begin{itemize}    
        \item 为了简单,所有的离线软件可以放置于一个仓库之中。
        \item 例如:LHCb的DIRAC项目,就是一个统一的仓库。里面包含不同的system。
              \url{https://github.com/DIRACGrid/DIRAC}。
        \item 在CMT和CVS或SVN结合时,用户可以checkout
              仓库中特定的目录或者文件。 我们不希望导出库中所有的源代码。
        \item Git由于分布式的原因,必须有导出所有的代码。
        \item 这里,经常困扰从cvs和svn迁移到git的用户。
        \item 由于我们的离线软件中,包含了各个方面的内容,因此,如果
              checkout所有的代码,可能有些庞大。另外,各个不同包之间的
              commit信息,是混合在一起的。这样可能效率低下。
        \item 因此,我们要根据需求,对这个离线软件进行拆分。
        \item Git提供了{\tt submodule}的特性,我们可以使用。
              \footnote{\scriptsize\url{http://git-scm.com/book/en/Git-Tools-Submodules}}
              \footnote{\scriptsize\url{http://git.wiki.kernel.org/index.php/GitSubmoduleTutorial}}
        \item 问题:如何拆分这些呢?
    \end{itemize}
\end{frame}

\begin{frame}
    \frametitle{目录层次 方案一}
    一个Container Package就对应一个Git仓库。
    \renewcommand*\DTstylecomment{\rmfamily\color{red}\textsc}
    \begin{block}{Offline Software}
    \dirtree{%
        .1 /\DTcomment{Git Repo}.
        .2 Simulation\DTcomment{Submodule}.
        .2 Reconstruction\DTcomment{Submodule}.
        .2 Calibration\DTcomment{Submodule}.
        .2 \ldots.
    }
    \end{block}

    \begin{block}{Simulation Software}
    \dirtree{%
        .1 Simulation\DTcomment{Git Repo}.
        .2 DetSimX.
        .2 GenSim.
        .2 PhysiSim.
        .2 PMTSim.
        .2 SimUtil.
    }
    \end{block}
\end{frame}

\begin{frame}
    \frametitle{方案一的优缺点}
    \begin{itemize}
        \item 层次比较单一。
        \item 需要建立的git仓库较少。
        \item 对于整个软件的发布较为简单。即Offline Software是由不同版本的
              Submodule(如Simulation, Reconstruction)等构建而成。
        \item 对于能否修改一个仓库下的Package,我们可以通过git服务端软件gitolite
              对用户进行限制。
        \item 开发人员导出代码时,可能还有很多无关的代码也被导出。
    \end{itemize}
\end{frame}

\begin{frame}
    \frametitle{目录层次 方案二}
    每个Package建立一个相应的GIT仓库。然后由Container Package包含这些仓库。
    而Container Package本身也是一个GIT仓库。
    \renewcommand*\DTstylecomment{\rmfamily\color{red}\textsc}
    \begin{block}{Offline Software}
    \dirtree{%
        .1 /\DTcomment{Git Repo}.
        .2 Simulation\DTcomment{Submodule}.
        .2 Reconstruction\DTcomment{Submodule}.
        .2 Calibration\DTcomment{Submodule}.
        .2 \ldots.
    }
    \end{block}

    \begin{block}{Simulation Software}
    \dirtree{%
        .1 Simulation\DTcomment{Git Repo}.
        .2 DetSimX\DTcomment{Submodule}.
        .2 GenSim\DTcomment{Submodule}.
        .2 PhysiSim\DTcomment{Submodule}.
        .2 PMTSim\DTcomment{Submodule}.
        .2 SimUtil\DTcomment{Submodule}.
    }
    \end{block}
\end{frame}

\begin{frame}
    \frametitle{方案二的优缺点}
    \begin{itemize}
        \item 层次比较复杂。
        \item 需要建立的git仓库会很多。在BOSS中,有三百多个包。
        \item 对于整个软件的发布较为复杂。即Offline Software是由不同版本的
              Submodule(如Simulation, Reconstruction)构建而成。
              而这些submodule又包含了不同版本的submodule。
        \item 开发人员导出代码时,可以导出特定的Package。
        \item 发布软件时,将会面对几百个包,要处理它们之间的版本。
    \end{itemize}
    \begin{block}{例子:Fedora Project Packages}
        \url{http://pkgs.fedoraproject.org/cgit/}
        \url{https://fedoraproject.org/wiki/Using\_Fedora\_GIT}
    \end{block}
\end{frame}

\begin{frame}
    \frametitle{目录层次 方案三}
    每个Package建立一个相应的GIT仓库。但是直接包含于Offline Software下。
    即Container Package直接存于Offline Software的仓库中。
    \renewcommand*\DTstylecomment{\rmfamily\color{red}\textsc}
    \begin{block}{Offline Software}
    \dirtree{%
        .1 /\DTcomment{Git Repo}.
        .2 Simulation.
        .3 DetSimX\DTcomment{Submodule}.
        .3 GenSim\DTcomment{Submodule}.
        .3 PhysiSim\DTcomment{Submodule}.
        .3 PMTSim\DTcomment{Submodule}.
        .3 SimUtil\DTcomment{Submodule}.
        .2 Reconstruction.
        .2 Calibration.
        .2 \ldots.
    }
    \end{block}
\end{frame}

\begin{frame}
    \frametitle{方案三的优缺点}
    \begin{itemize}
        \item 层次相对复杂。通过Offline Software的目录结构,
              可以了解可能包含的软件包。
        \item 用户在导出Offline Software这个项目后,
              可以使用{\tt git submodule}命令导出特定的包。
        \item 同样,需要建立的git仓库也很多。
        \item 发布软件时,也将会面对几百个包,要处理它们之间的版本。
    \end{itemize}
\end{frame}

\begin{frame}
    \frametitle{目录层次 方案X?}
    \begin{itemize}
        \item 对几个顶层的包分别建立git仓库,而这些包根据需要,
              考虑是否继续拆分?
        \item 如果包Private是私有的,仅用于辅助包Public,那么
              包Private就可以包含在包Public的仓库之中,而无需单独建立仓库。
        \item 但不管如何,还是需要先设计好基本的目录结构。
              按照CMT方式构建的软件需要有很好的逻辑关系。
        \item 尽量降低依赖关系,避免循环依赖。
        \item 而随着项目的进行,如果存在什么问题,我们需要及时改正。
        \item 有何建议?
    \end{itemize}
\end{frame}
